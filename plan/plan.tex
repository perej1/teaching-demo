\documentclass[a4paper,12pt]{article}
\usepackage[english]{babel}
\usepackage[utf8]{inputenc}
\usepackage{graphicx}
\usepackage{amsmath}
\usepackage{float}
\usepackage{amsmath,amssymb, mathtools}
\usepackage{listings}
\usepackage{siunitx}
\usepackage{color} %red, green, blue, yellow, cyan, magenta, black, white
\usepackage[toc,page]{appendix}
\usepackage{subfigure}
% Hyphenation
\usepackage[shortcuts]{extdash}
% Fonts
\usepackage{palatino}
%\usepackage{tgpagella}

% No indentation for paragraphs but space
%\usepackage{parskip}

\title{Plan for teaching experimentation: stationarity of a stochastic process}
\author{Jaakko Pere}
\date{\today}

\begin{document}
\maketitle

\paragraph{Intended learning outcomes.} The target group is a student in the
course \emph{MS-C2128 - Prediction and Time Series Analysis}. Preliminaries for
the course are some basic mathematics such as differential and integral
calculus, linear algebra and basic course in probability and statistics. Thus, I
think that even if the group of peers is heterogeneous, they should be able to
achieve the intended learning outcomes.

I wish to teach the concept of stationarity of a stochastic process. Rigorous
definitions are shown, but more importantly, I wish that after the session
students can recognize a stationary stochastic process when they see one. The
learning outcome is formulated in the following way:
\begin{itemize}
    \item After the session the student can recognize visually if a stochastic
    process is stationary.
\end{itemize}

\paragraph{Teaching methods} First, there is a presentation (lecture). I start by
figuring out the background knowledge of each student by asking some questions
like
\begin{itemize}
    \item Do you know what is a random variable?
    \item Do you know what is a stochastic process?
    \item Do you know what stationarity means in the context of stochastic
    processes?
\end{itemize}
Then I lecture. I use either blackboard or slides or maybe even both when
lecturing. The purpose is to give definitions and discuss them shortly.
Hopefully, the lecture will be more like a discussion with students, and I can
activate students enough during the lecture. So actually the ``lecture'' is a
combination of a traditional lecture and a teaching discussion.

After the lecture, there is some kind of assignment for students. I have not
decided yet what it is exactly, but I hope to code something with \textsf{R} so
that students can do the assignment with their laptops.

\paragraph{Motivation behind the teaching methods.} Firstly, the motivation for
lecture/discussion is that I have to somehow introduce the concepts to the
students. Secondly, the motivation for the assignment is that students get to
know in practice if they understand what is being taught. Also, the assignment
gives a possibility to correct misunderstandings if there are any.

\paragraph{Assessment.} The assignment should work as an assessment. If this was
not an experiment I would like to know how students did. However, I am not
yet sure if I want to somehow collect the results of the assignment or if it is
even practically possible.

\paragraph{Timeline.} While preparing the lecture and the assignment I have to
take into account that discussion (hopefully) takes some time and most probably
it takes some time to open the assignment on laptops.
\begin{itemize}
    \item 10min for lecturing/discussion.
    \item 10min for assignment.
\end{itemize}
\end{document}
